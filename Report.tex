%Version 2.1 April 2023
% See section 11 of the User Manual for version history
%
%%%%%%%%%%%%%%%%%%%%%%%%%%%%%%%%%%%%%%%%%%%%%%%%%%%%%%%%%%%%%%%%%%%%%%
%%                                                                 %%
%% Please do not use \input{...} to include other tex files.       %%
%% Submit your LaTeX manuscript as one .tex document.              %%
%%                                                                 %%
%% All additional figures and files should be attached             %%
%% separately and not embedded in the \TeX\ document itself.       %%
%%                                                                 %%
%%%%%%%%%%%%%%%%%%%%%%%%%%%%%%%%%%%%%%%%%%%%%%%%%%%%%%%%%%%%%%%%%%%%%

\documentclass[sn-basic,pdflatex]{sn-jnl}

%%%% Standard Packages
%%<additional latex packages if required can be included here>

\usepackage{graphicx}%
\usepackage{multirow}%
\usepackage{amsmath,amssymb,amsfonts}%
\usepackage{amsthm}%
\usepackage{mathrsfs}%
\usepackage[title]{appendix}%
\usepackage{xcolor}%
\usepackage{textcomp}%
\usepackage{manyfoot}%
\usepackage{booktabs}%
\usepackage{algorithm}%
\usepackage{algorithmicx}%
\usepackage{algpseudocode}%
\usepackage{listings}%
%%%%

%%%%%=============================================================================%%%%
%%%%  Remarks: This template is provided to aid authors with the preparation
%%%%  of original research articles intended for submission to journals published
%%%%  by Springer Nature. The guidance has been prepared in partnership with
%%%%  production teams to conform to Springer Nature technical requirements.
%%%%  Editorial and presentation requirements differ among journal portfolios and
%%%%  research disciplines. You may find sections in this template are irrelevant
%%%%  to your work and are empowered to omit any such section if allowed by the
%%%%  journal you intend to submit to. The submission guidelines and policies
%%%%  of the journal take precedence. A detailed User Manual is available in the
%%%%  template package for technical guidance.
%%%%%=============================================================================%%%%

%% Per the spinger doc, new theorem styles can be included using built in style, 
%% but it seems the don't work so commented below
%\theoremstyle{thmstyleone}%
\newtheorem{theorem}{Theorem}%  meant for continuous numbers
%%\newtheorem{theorem}{Theorem}[section]% meant for sectionwise numbers
%% optional argument [theorem] produces theorem numbering sequence instead of independent numbers for Proposition
\newtheorem{proposition}[theorem]{Proposition}%
%%\newtheorem{proposition}{Proposition}% to get separate numbers for theorem and proposition etc.

%% \theoremstyle{thmstyletwo}%
\theoremstyle{remark}
\newtheorem{example}{Example}%
\newtheorem{remark}{Remark}%

%% \theoremstyle{thmstylethree}%
\theoremstyle{definition}
\newtheorem{definition}{Definition}%



\raggedbottom




% tightlist command for lists without linebreak
\providecommand{\tightlist}{%
  \setlength{\itemsep}{0pt}\setlength{\parskip}{0pt}}





\begin{document}


\title[]{\textbf{A Predictive Modeling of Tracheostomy Readmissions}}

%%=============================================================%%
%% Prefix	-> \pfx{Dr}
%% GivenName	-> \fnm{Joergen W.}
%% Particle	-> \spfx{van der} -> surname prefix
%% FamilyName	-> \sur{Ploeg}
%% Suffix	-> \sfx{IV}
%% NatureName	-> \tanm{Poet Laureate} -> Title after name
%% Degrees	-> \dgr{MSc, PhD}
%% \author*[1,2]{\pfx{Dr} \fnm{Joergen W.} \spfx{van der} \sur{Ploeg} \sfx{IV} \tanm{Poet Laureate}
%%                 \dgr{MSc, PhD}}\email{iauthor@gmail.com}
%%=============================================================%%

\author[]{\fnm{Aabha} \sur{Latkar} \dgr{BA}}

\author[]{\fnm{Febriany} \sur{Lete} \dgr{SST}}

\author[]{\fnm{Maysen} \sur{Pagán} \dgr{BA}}



  \affil[]{\orgdiv{Master of Science in Statistical
Practice}, \orgname{Boston University}}

\abstract{\textbf{Purpose}: The abstract serves both as a general
introduction to the topic and as a brief, non-technical summary of the
main results and their implications. The abstract must not include
subheadings (unless expressly permitted in the journal's Instructions to
Authors), equations or citations. As a guide the abstract should not
exceed 200 words. Most journals do not set a hard limit however authors
are advised to check the author instructions for the journal they are
submitting to.

\textbf{Methods:} The abstract serves both as a general introduction to
the topic and as a brief, non-technical summary of the main results and
their implications. The abstract must not include subheadings (unless
expressly permitted in the journal's Instructions to Authors), equations
or citations. As a guide the abstract should not exceed 200 words. Most
journals do not set a hard limit however authors are advised to check
the author instructions for the journal they are submitting to.

\textbf{Results:} The abstract serves both as a general introduction to
the topic and as a brief, non-technical summary of the main results and
their implications. The abstract must not include subheadings (unless
expressly permitted in the journal's Instructions to Authors), equations
or citations. As a guide the abstract should not exceed 200 words. Most
journals do not set a hard limit however authors are advised to check
the author instructions for the journal they are submitting to.

\textbf{Conclusion:} The abstract serves both as a general introduction
to the topic and as a brief, non-technical summary of the main results
and their implications. The abstract must not include subheadings
(unless expressly permitted in the journal's Instructions to Authors),
equations or citations. As a guide the abstract should not exceed 200
words. Most journals do not set a hard limit however authors are advised
to check the author instructions for the journal they are submitting
to.\}}

\keywords{}



\maketitle

\hypertarget{sec1}{%
\section{Introduction}\label{sec1}}

\hypertarget{sec2}{%
\section{Data Preprocessing}\label{sec2}}

\hypertarget{sec3}{%
\section{Exploratory Data Analysis}\label{sec3}}

\hypertarget{sec4}{%
\section{Methods and Analysis}\label{sec4}}

\hypertarget{sec5}{%
\section{Conclusion}\label{sec5}}

\bmhead{Acknowledgments}

Acknowledgments are not compulsory. Where included they should be brief.
Grant or contribution numbers may be acknowledged.

Please refer to Journal-level guidance for any specific requirements.

\begin{appendices}

\hypertarget{secA}{%
\section{}\label{secA}}

An appendix contains supplementary information that is not an essential
part of the text itself but which may be helpful in providing a more
comprehensive understanding of the research problem or it is information
that is too cumbersome to be included in the body of the paper.

\end{appendices}

\bibliography{bibliography.bib}


\end{document}

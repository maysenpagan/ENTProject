%Version 2.1 April 2023
% See section 11 of the User Manual for version history
%
%%%%%%%%%%%%%%%%%%%%%%%%%%%%%%%%%%%%%%%%%%%%%%%%%%%%%%%%%%%%%%%%%%%%%%
%%                                                                 %%
%% Please do not use \input{...} to include other tex files.       %%
%% Submit your LaTeX manuscript as one .tex document.              %%
%%                                                                 %%
%% All additional figures and files should be attached             %%
%% separately and not embedded in the \TeX\ document itself.       %%
%%                                                                 %%
%%%%%%%%%%%%%%%%%%%%%%%%%%%%%%%%%%%%%%%%%%%%%%%%%%%%%%%%%%%%%%%%%%%%%

\documentclass[sn-basic,pdflatex]{sn-jnl}

%%%% Standard Packages
%%<additional latex packages if required can be included here>

\usepackage{graphicx}%
\usepackage{multirow}%
\usepackage{amsmath,amssymb,amsfonts}%
\usepackage{amsthm}%
\usepackage{mathrsfs}%
\usepackage[title]{appendix}%
\usepackage{xcolor}%
\usepackage{textcomp}%
\usepackage{manyfoot}%
\usepackage{booktabs}%
\usepackage{algorithm}%
\usepackage{algorithmicx}%
\usepackage{algpseudocode}%
\usepackage{listings}%
%%%%

%%%%%=============================================================================%%%%
%%%%  Remarks: This template is provided to aid authors with the preparation
%%%%  of original research articles intended for submission to journals published
%%%%  by Springer Nature. The guidance has been prepared in partnership with
%%%%  production teams to conform to Springer Nature technical requirements.
%%%%  Editorial and presentation requirements differ among journal portfolios and
%%%%  research disciplines. You may find sections in this template are irrelevant
%%%%  to your work and are empowered to omit any such section if allowed by the
%%%%  journal you intend to submit to. The submission guidelines and policies
%%%%  of the journal take precedence. A detailed User Manual is available in the
%%%%  template package for technical guidance.
%%%%%=============================================================================%%%%

%% Per the spinger doc, new theorem styles can be included using built in style, 
%% but it seems the don't work so commented below
%\theoremstyle{thmstyleone}%
\newtheorem{theorem}{Theorem}%  meant for continuous numbers
%%\newtheorem{theorem}{Theorem}[section]% meant for sectionwise numbers
%% optional argument [theorem] produces theorem numbering sequence instead of independent numbers for Proposition
\newtheorem{proposition}[theorem]{Proposition}%
%%\newtheorem{proposition}{Proposition}% to get separate numbers for theorem and proposition etc.

%% \theoremstyle{thmstyletwo}%
\theoremstyle{remark}
\newtheorem{example}{Example}%
\newtheorem{remark}{Remark}%

%% \theoremstyle{thmstylethree}%
\theoremstyle{definition}
\newtheorem{definition}{Definition}%



\raggedbottom




% tightlist command for lists without linebreak
\providecommand{\tightlist}{%
  \setlength{\itemsep}{0pt}\setlength{\parskip}{0pt}}





\begin{document}


\title[]{\textbf{A Predictive Modeling of Tracheostomy Readmissions}}

%%=============================================================%%
%% Prefix	-> \pfx{Dr}
%% GivenName	-> \fnm{Joergen W.}
%% Particle	-> \spfx{van der} -> surname prefix
%% FamilyName	-> \sur{Ploeg}
%% Suffix	-> \sfx{IV}
%% NatureName	-> \tanm{Poet Laureate} -> Title after name
%% Degrees	-> \dgr{MSc, PhD}
%% \author*[1,2]{\pfx{Dr} \fnm{Joergen W.} \spfx{van der} \sur{Ploeg} \sfx{IV} \tanm{Poet Laureate}
%%                 \dgr{MSc, PhD}}\email{iauthor@gmail.com}
%%=============================================================%%

\author[]{\fnm{Aabha} \sur{Latkar} \dgr{BA}}

\author[]{\fnm{Febriany} \sur{Lete} \dgr{SST}}

\author[]{\fnm{Maysen} \sur{Pagán} \dgr{BA}}



  \affil[]{\orgdiv{Master of Science in Statistical
Practice}, \orgname{Boston University}}

\abstract{Hospital readmissions following the completion of a procedure
poses significant challenges for both patients and hospitals. Being
readmitted could increase the risk of complications for patients and
introduce states of distress. For hospitals, readmissions present a
strain on resources and their reputation. This project aims to develop a
model that can predict whether a patient will be readmitted within 30
days of being discharged. After thorough data preprocessing and feature
extraction, three models were trained on patient data from 2018 and
tested on patient data from 2019 to determine predictive performance: a
null model, a support vector machine (SVM), and a random forest (RF).
Using Matthew's Correlation Coefficient to compare models, the HERE
model had the best performance with a coefficient of HERE while the null
model has a coefficient of HERE, and the HERE model has a coefficient of
HERE.}

\keywords{}



\maketitle

\hypertarget{sec1}{%
\section{Introduction}\label{sec1}}

In the medical field of Otolaryngology, preventing hospital readmissions
following procedures such as tracheostomies, total laryngectomies, or
mastoidectomies is significant both medically for patients as well as
financially for healthcare institutions. Medically, avoiding
readmissions can benefit patients' well-being as it reduces the possible
distress and suffering experienced from complications from new or
returning medical conditions. Financially, preventing readmissions is
crucial for hospitals who are paid by capitation. Capitation is a
payment system that pays hospitals a fixed amount per patient for a
prescribed period, therefore incentivizing hospitals to conduct less
procedures and treat patients as efficiently as possible. As a result,
hospitals paid by capitation incur the costs that are associated with
providing care to patients who are readmitted. Knowing if a patient
might be at higher risk of a readmission would allow doctors to increase
the effectiveness of their initial interventions and promote a smoother
recovery process while maintaining their reputation and quality of care.
Therefore, developing predictive models that can predict whether a
patient will be readmitted is essential for ensuring the efficiency of
healthcare. \newline \newline This project aims to build a model that
predicts if a patient who underwent a tracheostomy procedure is going to
be readmitted within 30 days of being discharged from the hospital. For
those patients who are readmitted within 30 days, this project also
analyzes the number of days until they will be readmitted as well as the
most common diagnoses that the patients will be readmitted with.

\hypertarget{sec2}{%
\section{Data Preprocessing}\label{sec2}}

\hypertarget{sec3}{%
\section{Exploratory Data Analysis}\label{sec3}}

\hypertarget{sec4}{%
\section{Methods and Analysis}\label{sec4}}

\hypertarget{sec5}{%
\section{Conclusion}\label{sec5}}

\bmhead{Acknowledgments}

Acknowledgments are not compulsory. Where included they should be brief.
Grant or contribution numbers may be acknowledged.

Please refer to Journal-level guidance for any specific requirements.

\begin{appendices}

\hypertarget{secA}{%
\section{}\label{secA}}

An appendix contains supplementary information that is not an essential
part of the text itself but which may be helpful in providing a more
comprehensive understanding of the research problem or it is information
that is too cumbersome to be included in the body of the paper.

\end{appendices}

\bibliography{bibliography.bib}


\end{document}

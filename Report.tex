%Version 2.1 April 2023
% See section 11 of the User Manual for version history
%
%%%%%%%%%%%%%%%%%%%%%%%%%%%%%%%%%%%%%%%%%%%%%%%%%%%%%%%%%%%%%%%%%%%%%%
%%                                                                 %%
%% Please do not use \input{...} to include other tex files.       %%
%% Submit your LaTeX manuscript as one .tex document.              %%
%%                                                                 %%
%% All additional figures and files should be attached             %%
%% separately and not embedded in the \TeX\ document itself.       %%
%%                                                                 %%
%%%%%%%%%%%%%%%%%%%%%%%%%%%%%%%%%%%%%%%%%%%%%%%%%%%%%%%%%%%%%%%%%%%%%

\documentclass[sn-basic,pdflatex]{sn-jnl}

%%%% Standard Packages
%%<additional latex packages if required can be included here>

\usepackage{graphicx}%
\usepackage{multirow}%
\usepackage{amsmath,amssymb,amsfonts}%
\usepackage{amsthm}%
\usepackage{mathrsfs}%
\usepackage[title]{appendix}%
\usepackage{xcolor}%
\usepackage{textcomp}%
\usepackage{manyfoot}%
\usepackage{booktabs}%
\usepackage{algorithm}%
\usepackage{algorithmicx}%
\usepackage{algpseudocode}%
\usepackage{listings}%
%%%%

%%%%%=============================================================================%%%%
%%%%  Remarks: This template is provided to aid authors with the preparation
%%%%  of original research articles intended for submission to journals published
%%%%  by Springer Nature. The guidance has been prepared in partnership with
%%%%  production teams to conform to Springer Nature technical requirements.
%%%%  Editorial and presentation requirements differ among journal portfolios and
%%%%  research disciplines. You may find sections in this template are irrelevant
%%%%  to your work and are empowered to omit any such section if allowed by the
%%%%  journal you intend to submit to. The submission guidelines and policies
%%%%  of the journal take precedence. A detailed User Manual is available in the
%%%%  template package for technical guidance.
%%%%%=============================================================================%%%%

%% Per the spinger doc, new theorem styles can be included using built in style, 
%% but it seems the don't work so commented below
%\theoremstyle{thmstyleone}%
\newtheorem{theorem}{Theorem}%  meant for continuous numbers
%%\newtheorem{theorem}{Theorem}[section]% meant for sectionwise numbers
%% optional argument [theorem] produces theorem numbering sequence instead of independent numbers for Proposition
\newtheorem{proposition}[theorem]{Proposition}%
%%\newtheorem{proposition}{Proposition}% to get separate numbers for theorem and proposition etc.

%% \theoremstyle{thmstyletwo}%
\theoremstyle{remark}
\newtheorem{example}{Example}%
\newtheorem{remark}{Remark}%

%% \theoremstyle{thmstylethree}%
\theoremstyle{definition}
\newtheorem{definition}{Definition}%



\raggedbottom




% tightlist command for lists without linebreak
\providecommand{\tightlist}{%
  \setlength{\itemsep}{0pt}\setlength{\parskip}{0pt}}





\begin{document}


\title[]{\textbf{A Predictive Modeling of Tracheostomy Readmissions}}

%%=============================================================%%
%% Prefix	-> \pfx{Dr}
%% GivenName	-> \fnm{Joergen W.}
%% Particle	-> \spfx{van der} -> surname prefix
%% FamilyName	-> \sur{Ploeg}
%% Suffix	-> \sfx{IV}
%% NatureName	-> \tanm{Poet Laureate} -> Title after name
%% Degrees	-> \dgr{MSc, PhD}
%% \author*[1,2]{\pfx{Dr} \fnm{Joergen W.} \spfx{van der} \sur{Ploeg} \sfx{IV} \tanm{Poet Laureate}
%%                 \dgr{MSc, PhD}}\email{iauthor@gmail.com}
%%=============================================================%%

\author[]{\fnm{Aabha} \sur{Latkar} \dgr{BSc}}

\author[]{\fnm{Febriany} \sur{Lete} \dgr{SST}}

\author[]{\fnm{Maysen} \sur{Pagán} \dgr{BA}}



  \affil[]{\orgdiv{Master of Science in Statistical
Practice}, \orgname{Boston University}}

\abstract{Hospital readmissions following the completion of a procedure
poses significant challenges for both patients and hospitals. Being
readmitted could increase the risk of complications for patients and
introduce states of distress. For hospitals, readmissions present a
strain on resources and their reputation. This project aims to develop a
model that can predict whether a patient will be readmitted within 30
days of being discharged. After thorough data preprocessing and feature
extraction, two models were trained on patient data from 2018 and tested
on patient data from 2019 to determine predictive performance: a support
vector machine (SVM) and a random forest (RF). Using Matthew's
Correlation Coefficient to compare models, the RF model had the best
performance with a coefficient of 0.75 while the SVM had a coefficient
of 0.61. From the RF model, Shapley values revealed that the all patient
refined diagnosis related group and number of procedures were the
features that contributed the most to the model's predictions.}

\keywords{}



\maketitle

\hypertarget{sec1}{%
\section{Introduction}\label{sec1}}

In the medical field of Otolaryngology, preventing hospital readmissions
following procedures such as tracheostomies, total laryngectomies, or
mastoidectomies is significant both medically for patients as well as
financially for healthcare institutions. Medically, avoiding
readmissions can benefit patients' well-being as it reduces the possible
distress and suffering experienced from complications from new or
returning medical conditions. Financially, preventing readmissions is
crucial for hospitals who are paid by capitation. Capitation is a
payment system that pays hospitals a fixed amount per patient for a
prescribed period, therefore incentivizing hospitals to conduct less
procedures and treat patients as efficiently as possible. As a result,
hospitals paid by capitation incur the costs that are associated with
providing care to patients who are readmitted. Knowing if a patient
might be at higher risk of a readmission would allow doctors to increase
the effectiveness of their initial interventions and promote a smoother
recovery process while maintaining their reputation and quality of care.
Therefore, developing predictive models that can predict whether a
patient will be readmitted is essential for ensuring the efficiency of
healthcare. \newline \newline This project aims to build a model that
predicts if a patient who underwent a tracheostomy procedure is going to
be readmitted within 30 days of being discharged from the hospital. For
those patients who are readmitted within 30 days, this project also
analyzes the number of days until they will be readmitted as well as the
most common diagnoses that the patients will be readmitted with. The
following analysis for this project was conducted on a sample of 3
million observations from the original data.

\hypertarget{sec2}{%
\section{Data Preprocessing}\label{sec2}}

The data for this project was provided to us from the Healthcare Cost
and Utilization Project (HCUP) Nationwide Readmissions Database (NRD).
This database provides information on admissions and discharges for
patients with and without repeat hospital visits within a given year. To
address the research questions of this project, we used data from the
years of 2018 and 2019, each of which had three data sets. For each
year, there was a \texttt{core} data set which contained main
information such as the age and sex of the patient as well as the
diagnoses and procedures the patient received for this given admission.
The \texttt{severity} data set provides information on the severity of
the condition for a given admission and the \texttt{hospital} data set
contains characteristics of the hospital the patient was admitted to
such as bed size.

\hypertarget{empty-diagnoses-and-procedures}{%
\subsection{Empty Diagnoses and
Procedures}\label{empty-diagnoses-and-procedures}}

The \texttt{core} data set contains 40 ICD-10-CM diagnosis code columns
representing 40 possible diagnoses a patient could receive at one
admission. Similarly, the data includes 25 ICD-10-PCS procedure code
columns representing 25 possible procedures a patient could receive at
one admission. One patient may not have 40 total diagnoses or 25 total
procedures and as a result, many of the cells in these columns are
empty. The first step in preparing the data for analysis was replacing
these empty cells with \texttt{NA}s which will make future filtering and
one-hot-encoding steps easier.

\hypertarget{creating-binary-readmitted-response-column}{%
\subsection{Creating Binary Readmitted Response
Column}\label{creating-binary-readmitted-response-column}}

The focus of this project is predicting readmissions for tracheostomy
procedures. Therefore, our next step was to filter the \texttt{core}
data set to include only those patients who received a tracheostomy
during their admission. The ICD-10-PCS codes for tracheostomies all
begin with ``0B11'' with other numbers and letters following for
different tracheostomy approaches such as an open approach or
percutaneous approach. We applied the filter function to the 25
procedure columns to obtain all patients who had an ICD-10-PCS code that
began with ``0B11'' in any one of the 25 possible columns. Extracting
the unique IDs or Visit Links from this filtered set and filtering
\texttt{core} to get patients only with those Visit Links allowed us to
create a new data frame that contained all patients who were admitted
for a tracheostomy and all other admissions by that patient. A binary
column was then added to this new data frame which was a 1 if the
patient received a tracheostomy at the corresponding visit and 0
otherwise. For those tracheostomy admissions, we added the length of
stay (LOS) variable to the days to event variable which gives us a
sequence of numbers representing the ``date'' the patient was
discharged. This data frame contains all patient admissions both before
and after they received the tracheostomy. As a result, for each patient,
we then filtered the data frame again to include only the tracheostomy
admission and all admissions that occurred after that visit. \newline
\newline This new data frame now allows us to create our binary response
variable that is 1 if the patient was readmitted to a hospital within 30
days of being discharged from a tracheostomy procedure. If a patient had
no admissions following their tracheostomy admission, the readmitted
response variable was 0. If subtracting the days to event value of the
next admission after the tracheostomy from the days to event value of
the tracheostomy admission results in a value less than or equal to 30,
the readmitted response variable was 1. If the difference of the two
days to event values was greater than 30, the readmitted response
variable was 0. \newline \newline We then removed all rows that
represented hospital admissions after the tracheostomy to obtain a data
frame where each row represented a unique patient's tracheostomy
admission along with the binary response variable indicating if they
were readmitted within 30 days of the procedure.

\hypertarget{right-censoring}{%
\subsection{Right Censoring}\label{right-censoring}}

Right censored data is used to describe data where subjects leave the
study before an event occurring or the study ends before the event has
occurred. In the context of this project, right censoring can occur if a
patient received a tracheostomy in December of one year and was
readmitted in January of the following year. The data from the NRD is
constructed in such a way that it is difficult to identify patients. The
patient IDs or Visit Links change from one year to the next for the same
patient. As a result, if a patient were to receive a tracheostomy in
December of 2018, we would be unable to determine and therefore predict
if the patient was readmitted within 30 days since it is unknown if they
were readmitted in January of 2019. To solve this issue, we make our
window of inclusion for patients between January and November of a given
year by filtering out those patients who were admitted for a
tracheostomy in December.

\hypertarget{one-hot-encoding}{%
\subsection{One-Hot-Encoding}\label{one-hot-encoding}}

When creating the predictive model for readmissions, we wish to conduct
comorbidity analysis which will allow us to include the presence of
certain diagnoses as predictor variables. For that reason, we will need
to one-hot-encode all ICD-10-CM diagnosis codes. For each unique
diagnosis code in the 40 diagnosis columns, we added a column for that
code which equaled 1 if the patient received the diagnosis on their
tracheostomy admission and equaled 0 otherwise.

\hypertarget{joining-data-sets}{%
\subsection{Joining Data Sets}\label{joining-data-sets}}

The last step in preparing the data for modeling and analysis was to
join the three data sets. Each data set had a NRD record identifier
variable which gave a unique code to each hospital admission. Therefore,
to our cleaned \texttt{core} data set, we joined the \texttt{severity}
and \texttt{hospital} data sets by the record identifier.

\hypertarget{sec3}{%
\section{Exploratory Data Analysis}\label{sec3}}

Looking at Figure~\ref{fig:CIB} which displays the counts of readmitted
and non-readmitted patients, we notice that we have a class imbalance in
our data set, where the number of instances labeled as '1's
significantly exceeds those labeled as '0's. This may potentially lead
to model biases, poor generalization, and challenges in decision-making
during training and evaluation. \newline \newline In this specific
situation, due to resource constraints and time limitations, we
prioritized other aspects of the analysis pipeline that we deemed more
critical for achieving our project objectives. However, we acknowledge
the importance of addressing this imbalance in future iterations to
enhance model robustness and accuracy.\newline \newline From
Figures~\ref{fig:AGE} and \ref{fig:AGE2} the age distributions indicate
a right skew, suggesting that older patients are more inclined to
undergo tracheotomies. Notably, the median age for patients with
readmissions is slightly lower compared to those without readmission.
While the distributions for both groups share similarities, the peak in
the distribution for patients with readmissions appears sharper,
indicating a more concentrated age group, whereas the distribution for
patients without readmission displays a broader spread across different
age groups.\newline \newline In Figure~\ref{fig:APRDRG}, we observe
variations in readmission proportions across different all patient
refined diagnosis related groups (DRG) categories. All patient refined
DRGs are broader categories of diagnosis that include the severity of
the diagnosis as well. Some all patient refined DRG categories exhibit a
notably higher proportion of readmissions compared to others within the
sample. Consequently, we opted to incorporate all patient refined DRG as
a predictor in our analysis to account for this observed variance.

\hypertarget{sec4}{%
\section{Methods and Analysis}\label{sec4}}

In this section, we address and analyze the three objectives of this
project.

\hypertarget{sec4A}{%
\subsection{Predictive Models}\label{sec4A}}

The main objective of this project is to predict whether or not a
patient will return to the hospital within 30 days of being discharged
from a tracheostomy procedure. This is a classification problem which
influences us to apply machine learning classification models. We look
at two such models: a support vector machine and a random forest.

\hypertarget{sec4AA}{%
\subsubsection{Comorbidity Analysis}\label{sec4AA}}

Before fitting the two models, we first consider what predictors to
include. With over 500 possible diagnosis codes each patient can
receive, we choose to select a subset of these codes that we believe
would have the largest influence on predicting if a patient would be
readmitted. To do this, we first observed the proportion of readmitted
patients compared to the proportion of non-readmitted patients who
received each comorbidity. Here, comorbidity is defined as any diagnosis
code that a patient was diagnosed with at the same visit as their
tracheostomy procedure. We plotted these proportions on a bar plot in
descending order in hopes of being able to compare the proportions
between both groups and determine which codes to include in our model
based on the largest differences. This plot can be viewed in
Figure~\ref{fig:props}. \newline \newline However, with such a large
number of unique codes, the proportions are quite small and hard to
compare. As a result, we then plotted the difference between the
proportions of both groups for each comorbidity. In other words, for
each comorbidity we plotted the absolute difference between the
proportion of readmitted patients who were diagnosed with that
comorbidity and the proportion of non-readmitted patients who were
diagnosed with that comorbidity. We then chose a cutoff value of 0.015
which returned 40 comorbidities whose difference in proportions was
greater than or equal to 0.015. This plot can be viewed in
Figure~\ref{fig:diffs}. The top two diagnosis codes with the largest
difference in proportions for readmitted and non-readmitted patients is
Z515 and J9601 which represent encounter for palliative care and acute
respiratory failure with hypoxia respectively. \newline \newline We also
included the following variables that we believe to have an association
with whether or not a patient is readmitted: age, sex, number of
diagnoses, number of procedures, all patient refined DRG, and hospital
bed size.

\hypertarget{sec4AB}{%
\subsubsection{Support Vector Machine}\label{sec4AB}}

The first model we chose to fit is a support vector machine (SVM) which
finds a hyperplane to separate our data into two classes (readmitted and
non-readmitted) as well as possible while allowing for some violations
to this separation to prevent overfitting. The SVM uses what is called a
kernel to quantify the similarity between observations. The most popular
kernels are the linear kernel, polynomial kernel, and radial kernel with
the polynomial and radial kernels allowing for a more flexible decision
boundary. Therefore, on our 2018 training data, we fit three SVMs, one
for each kernel using the predictors mentioned in Section~\ref{sec4AA}.
We then plotted the ROC curves to evaluate the performance of each
classification model on our 2019 testing data. Figure~\ref{fig:svm_roc}
shows that comparing these three kernels, the linear kernel performs the
best with an area under the curve (AUC) of 0.756. Therefore, we continue
our analysis with the SVM with linear kernel.

\begin{figure}[H]

{\centering \includegraphics[width=0.8\linewidth]{figures/svm_roc} 

}

\caption{Plot of the ROC curves for SVMs with linear, polynomial and radial kernels.}\label{fig:svm_roc}
\end{figure}

Figure~\ref{fig:svm_conf} below displays the confusion matrix of the SVM
with linear kernel model. Due to the imbalanced nature of our data where
the readmitted class is larger than the non-readmitted class, the metric
of accuracy may be misleading. For example, if 90\% of our data belongs
to the readmitted class and only 10\% belongs to the non-readmitted
class, a model that predicts all observations to belong to the
readmitted class can achieve 90\% accuracy. As a result, to evaluate how
well our model correctly predicts observations, we turn to Matthew's
Correlation Coefficient (MCC) which takes into account true positives
(TP), true negatives (TN), false positives (FP), and false negatives
(FN). The MCC of our SVM model is 0.6176 and the formula is: \newline \[
MCC = \frac{(TP*TN) - (FP*FN)}{\sqrt{(TP+FP)(TP+FN)(TN+FP)(TN+FN)}}
\]

\begin{figure}[H]

{\centering \includegraphics[width=0.25\linewidth]{figures/svm_conf} 

}

\caption{Confusion matrix for SVM with linear kernel.}\label{fig:svm_conf}
\end{figure}

\hypertarget{sec4AC}{%
\subsubsection{Random Forest}\label{sec4AC}}

The second model we considered was a Random Forest (RF) model which
takes our high dimensional data and aggregates decision trees to improve
overall predictive performance. Figure~\ref{fig:rf_conf} displays the
confusion matrix for the RF and the MCC can be calculated to get 0.7489.

\begin{figure}[H]

{\centering \includegraphics[width=0.25\linewidth]{figures/rf_conf} 

}

\caption{Confusion matrix for RF.}\label{fig:rf_conf}
\end{figure}

We observe that the RF MCC is over 20\% larger than the SVM MCC
indicating that the RF is the better model in predicting whether or not
a patient is readmitted within 30 days of their tracheostomy discharge.

\hypertarget{sec4ACA}{%
\subsubsection{Feature Importance}\label{sec4ACA}}

From our best model, we wish to determine which features contribute the
most to make predictions using Shapley values. Each classification
prediction can be broken down into a sum of contributions from each
input variable of our RF model. These contributions are the Shapley
values. We can observe the mean absolute Shapley value for each feature
across all the data to determine each feature's overall contribution
towards the prediction. The top 10 features can be view in
Figure~\ref{fig:rf_imp}.

\begin{figure}[H]

{\centering \includegraphics[width=0.75\linewidth]{figures/rf_imp} 

}

\caption{Feature importance plot of RF model showing the mean absolute Shapley values for each predictor.}\label{fig:rf_imp}
\end{figure}

Features with higher Shapley values are considered more important as
they have a greater impact on the model's predictions. From this figure,
the all patient refined DRG marginally contributes the most to the RF's
predictions followed by the number of procedures the patient received
during their tracheostomy visit as well as whether or not the patient
received the diagnosis code Z931 for gastrostomy status.

\hypertarget{sec4B}{%
\subsection{Days to Readmission}\label{sec4B}}

Another objective of this project was to explore the number of days
until patients were readmitted for those who were readmitted within 30
days. Figure~\ref{fig:daysto} plots the counts of days to readmission
comparing males and females.

\begin{figure}[H]

{\centering \includegraphics[width=0.9\linewidth]{figures/daysto} 

}

\caption{Barplot of proportion of males and proportion of females who were readmitted to a hospital after a specific number of days following discharge.}\label{fig:daysto}
\end{figure}

From this figure, we can see that most readmitted patients return to the
hospital within 10 days of being discharged from the tracheostomy. It is
interesting to note that while the largest proportion of males are
readmitted after 1 day, there is an equal proportion of females who are
readmitted after 1 day as well as after 20 days of being discharged.

\hypertarget{sec4C}{%
\subsection{Readmitted Diagnoses}\label{sec4C}}

We also created a similar figure looking at the most common diagnoses
that patients were readmitted with. For those patients who were
readmitted within 30 days of their tracheostomy discharge, we calculated
the proportion of males and the proportion of females who received a
particular diagnosis code at their readmitted visit. We then plotted
these proportions and used the proportion cutoff of 0.15 to observe the
top 15 most common readmission diagnoses.\newline \newline
Figure~\ref{fig:readmit_diags} shows that the most common readmitted
diagnosis, with an equal proportion of males and females receiving it,
is I10 which represents essential (primary) hypertension. This is
followed by A419 which is sepsis, a condition in which the body responds
improperly to an infection. Similar to hypertension, an equal proportion
of males and females are readmitted with sepsis. \newline \newline The
figure also shows that there are 4 common readmission diagnoses that
females are diagnosed with but males are not. These codes are F419,
E876, F329, and N390 which represent anxiety disorder, hypokalemia,
single episode major depressive disorder, and urinary tract infection
respectively. Within these top 15 most common readmission diagnoses,
there is one code that males are diagnosed with and females are not: E43
representing unspecified severe protein-calorie malnutrition.

\hypertarget{sec5}{%
\section{Conclusion}\label{sec5}}

In our study, we conducted extensive data preprocessing which
encompassed various tasks such as handling empty diagnosis and procedure
cells, creating a binary readmitted response, addressing right
censoring, and executing one-hot encoding of diagnoses codes. During
exploratory data analysis, we observed that a majority of readmitted
patients revisited the hospital within a 30-day window, particularly in
the age range of 55-75. We also noticed some all patient refined DRGs
will influence the readmitted or non-readmitted prediction. Moving to
data analysis, comorbidities were included as predictors. Through the
application of SVM and RF algorithms, we determined that a RF yielded
better predictions of readmission, as evidenced by the MCC. Further
analysis revealed that a considerable portion of readmitted patients
returned within 10 days post-discharge, with distinct readmission trends
between genders. Notably, the highest proportion of male patients was
readmitted within one day, while females also exhibited a peak around
day 20. Additionally, we identified prevalent diagnoses among readmitted
patients, including essential hypertension and sepsis, alongside
gender-specific conditions like anxiety disorder and urinary tract
infections. \newline \newline Future work on this project would first
include incorporating the full data set rather than a sample of 3
million. Applying our data preprocessing and models to the full data set
would improve the model's generalizability. We could also apply
one-hot-encoding to the various procedures that patients received along
with their tracheostomy. Conducting a similar analysis like we did with
comorbidities would allow us to add binary features for receiving
particular procedures to the model. \newline \newline Comprehending
these analyses holds significance for doctors in shaping early
intervention strategies, refining patient care, and optimizing resource
allocation, particularly concerning the tracheostomy procedure. By
understanding the factors influencing these readmission patterns,
healthcare professionals can proactively identify at-risk patients,
intervene promptly, and tailor treatment plans to mitigate the
likelihood of readmission. \newpage \bmhead{Acknowledgments}

We thank Dr.~Anand K. Devaiah, Dean Kennedy, and Dr.~Jessica R. Levi
from the Boston University Chobanian \& Avedisian School of Medicine for
providing the objectives of this project, access to the data, and
assistance with any questions we had throughout the project. We also
thank Professor Masanao Yajima from Boston University MSSP for his
guidance, support, and feedback throughout this process.

\bmhead{Authors' Contributions}

MP, AL, and FL wrote code for data cleaning, AL wrote code for and
conducted exploratory data analysis, MP and AL wrote code for
one-hot-encoding, MP conducted comorbidity analysis, FL wrote code for
predictive models, MP and AL explored project objectives two and three.

\begin{appendices}

\hypertarget{secA}{%
\section{}\label{secA}}

\begin{figure}[H]

{\centering \includegraphics[width=1\linewidth]{figures/CIB} 

}

\caption{ Counts of Readmissions in sample of 3 million.}\label{fig:CIB}
\end{figure}
\begin{figure}[H]

{\centering \includegraphics[width=1\linewidth]{figures/AGE} 

}

\caption{ Distribution of Patient Ages by Readmission Status after Tracheostomy}\label{fig:AGE}
\end{figure}
\begin{figure}[H]

{\centering \includegraphics[width=1\linewidth]{figures/AGE2} 

}

\caption{ Median Ages of Patients by Readmission Status after Tracheostomy }\label{fig:AGE2}
\end{figure}
\begin{figure}[H]

{\centering \includegraphics[width=1\linewidth]{figures/APRDRG} 

}

\caption{Proportion of Readmission Among APRDRG categories in the sample}\label{fig:APRDRG}
\end{figure}
\begin{figure}[H]

{\centering \includegraphics[width=0.9\linewidth]{figures/readmit_diags} 

}

\caption{Barplot of proportion of males and proportion of females who received specific diagnosis codes at their readmission visit.}\label{fig:readmit_diags}
\end{figure}
\begin{figure}[H]

{\centering \includegraphics[width=1\linewidth]{figures/props} 

}

\caption{Barplot of proportion of readmitted patients and proportion of non-readmitted patients who received specific diagnosis codes at their tracheostomy visit.}\label{fig:props}
\end{figure}
\begin{figure}[H]

{\centering \includegraphics[width=1\linewidth]{figures/diffs} 

}

\caption{Barplot of proportion differences between readmitted patients and non-readmitted patients who received specific diagnosis codes at their tracheostomy visit.}\label{fig:diffs}
\end{figure}

\end{appendices}



\end{document}
